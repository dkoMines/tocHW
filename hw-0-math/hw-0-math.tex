\documentclass[12pt,letterpaper]{ntdhw}


\usepackage{ntdmath}
\usepackage{ntdcourse}

\title{Homework: Math Preliminaries and Notation}
\author{\coursecode}

\begin{document}
\pagestyle{fancyplain}


\rhead{Name:\hspace{1.5in}}

\maketitle
\thispagestyle{fancyplain}

\noindent { \em Instructions: Answer the following questions.  You are
  encouraged to use the provided {\LaTeX} source to edit and submit
  your answers.  }

% \clearpage

\begin{enumerate}
  \item Set Builder Notion: Find the elements of the following sets
  \begin{enumerate}
     \item $a = \setbuilder{x \in \intset}{x \textrm{ is even and between
         0 and 5, exclusive}} $


     \item $a = \setbuilder{(x \in \intset,y \in \intset)}{\left(0 \leq x < 2\right) \land
       \left(0 \leq y < 2\right)} $


  \end{enumerate}

  \item Set Operations
  \begin{description}
    \item[Given] \descnl
    \begin{itemize}
      \item $a = \cbrace{1,2,3,5,7}$
      \item $b = \cbrace{1,3,5,7,9}$
      \item $c = \cbrace{2,4}$
    \end{itemize}
    \item[Find] \descnl
    \begin{enumerate}
      \item $a \cup b$
      \item $b \cup a$
      \item $a \cap b$
      \item $b \cap a$
      \item $a \setminus b$
      \item $b \setminus a$
      \item $a \times c$
      \item $c \times a$
      \item $\powerset{c}$ =
      \end{enumerate}
  \end{description}

  \item Boolean Algebra: Simplify and write in terms of $a$, $b$, AND,
  OR, NOT, 0, and 1.
  \begin{enumerate}
    \item $a \lor \lnot a$
    \item $a \land \lnot a$
    \item $\lnot (a \iff b)$
  \end{enumerate}


  \item Use a truth table to prove the distributivity or $\lor$ over
  $\land$:\\
  $a \lor (b \land c) = (a \lor b) \land (a \lor c)$.

  \item Set Algebra: Simplify
  \begin{enumerate}
    \item $a \cup a$
    \item $a \cap \overline{a}$
    \item $a \cup (a \cap b)$
  \end{enumerate}

  \item Set Algebra: Always true, always false, or unknown (i.e., depends on the
  values of $a$ and $b$).
  \begin{enumerate}
    \item $a \setminus b \subseteq a$
    \item $a \cap b \subseteq a$
    \item $a \cup b \subseteq a$
    \item $\cbrace{1,2} = \cbrace{2,1}$
    \item $\pbrace{1,2} = \pbrace{2,1}$
  \end{enumerate}

  \item Function Notion: Translate the following C function prototypes
  to mathematical function notation.  Assume that the C {\tt int} type
  corresponds to mathematical integers and the C {\tt float} type
  corresponds to mathematical reals.  \emph{(Hint: You may find the
    {\tt cdecl} program useful to explain the C syntax.  See {\tt man
      cdecl} for usage details.)}
  \begin{enumerate}
    \item {\tt int f(int, int)}
    \item {\tt float g(\_Bool, float)}

    \item {\tt int h(int(*)(int,int),int)}
  \end{enumerate}

  \item Prove by induction that merge-sort is correct, i.e., it
  returns the sorted sequence.

\end{enumerate}

\end{document}

%%% Local Variables:
%%% mode: latex
%%% TeX-master: t
%%% End:
